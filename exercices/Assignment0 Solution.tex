\documentclass[a4paper,oneside,12pt]{article}
\usepackage{graphics,multicol}
\usepackage[latin1]{inputenc}
\usepackage[a4paper,margin=2cm,lmargin=2cm,headheight=1pt]{geometry}
\usepackage{fancyhdr}
\usepackage{amsmath,amssymb,amsfonts,amsthm, mathrsfs}
\usepackage{mathtools,bbm}
\usepackage{IEEEtrantools}
\usepackage[]{algorithm2e}
\usepackage{tikz,pgfplots,tkz-graph}
\usepackage{ifthen,calc}
\usepackage{listings,hyperref,float}
\usepackage{epsfig,epstopdf,graphicx}
\usepackage{enumerate}

\pgfplotsset{compat=newest}
\usetikzlibrary{positioning,matrix}
\interdisplaylinepenalty=2500

% Page settings
% \setlength{\oddsidemargin}{0 in}
% \setlength{\evensidemargin}{0 in}
% \setlength{\topmargin}{-0.6 in}
% \setlength{\textwidth}{6.5 in}
% \setlength{\textheight}{8.5 in}
% \setlength{\headsep}{0.75 in}
% \setlength{\parindent}{0 in}
% \setlength{\parskip}{0.1 in}

% Markov symbol X -o- Y
\newcommand{\markov}{\mathrel{\multimap}\joinrel\mathrel{-}\joinrel\mathrel{\mkern-6mu}\joinrel\mathrel{-}}
% Independent & not independent symbol X _||_ Y
\newcommand{\indep}{\mathrel{\bot}\joinrel\mathrel{\mkern-5mu}\joinrel\mathrel{\bot}}
\newcommand{\dep}{\centernot\indep}
% i.i.d.
\newcommand{\iid}{\stackrel{\mathrm{i.i.d.}}{\sim}}
% Integration d
\newcommand{\dd}{\mathop{}\!\mathrm{d}}

% Single braces
\newcommand{\agbrs}[1]{\left\langle #1 \right\rangle}	% angle braces,  	< x >
\newcommand{\rdbrs}[1]{\left( #1 \right)}				% round braces,  	( x )
\newcommand{\sqbrs}[1]{\left\lbrack #1 \right\rbrack}	% square braces, 	[ x ]
\newcommand{\clbrs}[1]{\left\lbrace #1 \right\rbrace}	% curly braces, 	{ x }

% Delimited braces
\newcommand{\agbrsv}[2]{\left\lange #1 \,\middle|\, #2 \right\rangle}	% angle braces with delimiter |,  	< x | y >, inner product
\newcommand{\rdbrsv}[2]{\left( #1 \,\middle|\, #2 \right)}				% round braces with delimiter |,  	( x | y ), conditional probability
\newcommand{\sqbrsv}[2]{\left\lbrack #1 \,\middle|\, #2 \right\rbrack}	% square braces with delimiter |,  	[ x | y ], conditional expectation
\newcommand{\clbrsv}[2]{\left\lbrace #1 \,\middle|\, #2 \right\rbrace}	% curly braces with delimiter |,  	{ x | y }, set

% Short for underline and overline
\newcommand{\udl}[1]{\underline{#1}}
\newcommand{\ovl}[1]{\overline{#1}}

% Derivative and partial derivation
\newcommand{\fracd}[2]{ \frac{\mathrm{d} #1}{\mathrm{d} #2} }
\newcommand{\fracp}[2]{ \frac{\partial #1}{\partial #2} }

% Absolute value, norm, ceil, floor, and evaluation
\newcommand{\abs}[1]{\left| #1 \right|}		% Absolute value, 	| x |
\newcommand{\norm}[1]{\left\| #1 \right\|}	% Norm, 			|| x ||
\DeclarePairedDelimiter\ceil{\lceil}{\rceil}
\DeclarePairedDelimiter\floor{\lfloor}{\rfloor}
\newcommand{\eval}[1]{\left. #1 \right|}	% Evaluation, 		f(x) |_{x=x0}

% Definition equal sign
\newcommand{\defeq}{\vcentcolon=}   % :=
\newcommand{\eqdef}{=\vcentcolon}   % =:
\newcommand{\texteq}[1]{\stackrel{#1}{=\joinrel=\joinrel=}}   % use for change of variable

% Inverse hyperbolic functions
\DeclareMathOperator\arcsinh{arcsinh}
\DeclareMathOperator\arccosh{arccosh}
\DeclareMathOperator\arctanh{arctanh}
\DeclareMathOperator\atanh{atanh}
\DeclareMathOperator\sech{sech}

% argmax and argmin
\newcommand\limit[1]{\underset{#1}{\lim}\,}
\newcommand\argmax[1]{\mathrm{arg}\,\underset{#1}{\max}\,}
\newcommand\argmin[1]{\mathrm{arg}\,\underset{#1}{\min}\,}

% Matrices
\newcommand*\pmtx[1]{\begin{pmatrix}#1\end{pmatrix}}			% Matrix with round braces
\newcommand*\bmtx[1]{\begin{bmatrix}#1\end{bmatrix}}			% Matrix with square braces
\newcommand*\vmtx[1]{\begin{vmatrix}#1\end{vmatrix}}			% Determinant of matrix
\newcommand*\spmtx[1]{\rdbrs{\begin{smallmatrix}#1\end{smallmatrix}}}	% Small matrix with round braces
\newcommand*\sbmtx[1]{\sqbrs{\begin{smallmatrix}#1\end{smallmatrix}}}	% Small matrix with square braces
\newcommand*\svmtx[1]{\abs{\begin{smallmatrix}#1\end{smallmatrix}}}		% Determinant of small matrix
\newcommand{\sizecorr}[1]{\makebox[0cm]{\phantom{$\displaystyle #1$}}}	% Get size of an expression

% Theorem environments
\newtheorem{theorem}{Theorem}
\newtheorem{corollary}[theorem]{Corollary}
\newtheorem{lemma}[theorem]{Lemma}
\newtheorem{observation}[theorem]{Observation}
\newtheorem{proposition}[theorem]{Proposition}
\newtheorem{definition}[theorem]{Definition}
\newtheorem{claim}[theorem]{Claim}
\newtheorem{fact}[theorem]{Fact}
\newtheorem{assumption}[theorem]{Assumption}

\newtheorem*{theorem*}{Theorem}
\newtheorem*{corollary*}{Corollary}
\newtheorem*{lemma*}{Lemma}
\newtheorem*{proposition*}{Proposition}
\newtheorem*{definition*}{Definition}
\newtheorem*{example*}{Example}
\newtheorem*{remark*}{Remark}
\newtheorem*{problem*}{Problem}

% Solution environment
\newenvironment{solution}{\begin{proof}[Solution]}{\end{proof}}

% Short for boldsymbol
\newcommand{\bs}[1]{\boldsymbol{#1}}

% Other frequent used operators
\def\Var{\mathsf{Var}}
\def\Cov{\mathsf{Cov}}
\def\coeff{\mathsf{coeff}}
\def\Tr{\mathsf{Tr}}
\def\rank{\mathsf{rank}}
\def\diag{\mathsf{diag}}
\def\mse{\mathsf{mse}}
\def\mmse{\mathsf{mmse}}

\def\ee{\mathbbm{e}}
\def\ii{\mathbbm{i}}
\def\EE{\mathsf{E}}
\def\FF{\mathsf{F}}
\def\SS{\mathsf{S}}
\def\UU{\mathsf{U}}
\def\xx{\mathsf{x}}
\def\pprime{{\prime\prime}}

\newcommand{\KL}[2]{D\left( #1 \,\middle|\middle|\, #2 \right)}%                   ( | )

\makeatletter
\newcommand\ztag[1]{%
\def\@currentlabel{#1}%
\gdef\tmp{%
\addtocounter{equation}{-1}%
\def\theequation{#1}}%
\aftergroup\aftergroup\aftergroup\aftergroup\aftergroup\aftergroup
\aftergroup\aftergroup\aftergroup\aftergroup\aftergroup\aftergroup
\aftergroup\aftergroup\aftergroup\aftergroup\aftergroup\aftergroup
\aftergroup\aftergroup\aftergroup\aftergroup\aftergroup\aftergroup
\aftergroup\aftergroup\aftergroup\aftergroup\aftergroup\aftergroup
\aftergroup
\tmp\IEEEyesnumber}


\pagestyle{empty}
\begin{document}

\noindent

\begin{tabular}{lcr}
  Duke University & & Math 690-40 \\  
  Homework zero & \hspace{6.3cm} & F. Krzakala and L. Zdeborova\\ \hline
\end{tabular}

\begin{center}
  {\Large {\bf Graphical models and the Ising model}}
\end{center}

\subsection*{Representing problems by graphical models}

Write the following problems (a) in terms of a probability distribution and (b) in terms of a graphical model by drawing an example of the corresponding factor graph. 

\begin{itemize}
\item[(1)]{\bf p-spin model} 

    One model that is commonly studied in physics is the so-called Ising $3$-spin
    model. The Hamiltonian of this model is written as 
    \begin{equation}
        {\cal H} (\{S_i \}_{i=1}^N) = - \sum_{(ijk)\in E} J_{ijk} S_i S_j S_k  - \sum_{i=1}^N h_i S_i
    \end{equation}
    where $E$ is a given set of (unordered) triplets $i \neq j \neq k$,
    $J_{ijk}$ is the interaction strength for the triplet $(ijk)\in E$, and
    $h_i$ is a magnetic field on spin $i$. The spins are Ising, which in
    physics means $S_i \in \{+1, -1 \}$.  

    \begin{solution} $\,$ 
    \begin{enumerate}[(a)]
    \item   
            The corresponding Boltzmann distribution of the given model is
            \begin{IEEEeqnarray}{rCl}
                P( \clbrs{ S_i }_{i=1}^N )
                &=& \frac{1}{Z(\beta)} \exp \rdbrs{ -\beta \mathcal{H} ( \clbrs{ S_i }_{i=1}^N )} \IEEEnonumber \\
                &=& \frac{1}{Z(\beta)} \sqbrs{ \prod_{(ijk) \in E} \ee^{\beta J_{ijk} S_i S_j S_k} } \sqbrs{ \prod_{i=1}^N \ee^{\beta h_i S_i} }, \label{eq:1_1}
            \end{IEEEeqnarray}
            where $ \beta $ is the inverse temperature and
            \begin{equation*}
                  Z(\beta) = \sum_{\clbrs{ S_i }_{i=1}^N} \exp \rdbrs{ -\beta \mathcal{H} ( \clbrs{ S_i }_{i=1}^N )},
            \end{equation*} 
            is the partition function under inverse temperature $ \beta $ to guarantee that $ P( \clbrs{ S_i }_{i=1}^N ) $ sum to one for all possible configurations.
    \item   
            From eqn~\ref{eq:1_1} we can see the probability distribution (without normalization) is the product of $ \abs{E} $ interaction terms and $ N $ local magnetic field terms.

            For example, consider a $ 3 $-spin model with $ N = 6 $ spins and triplet set $ E = \clbrs{ (123), (245), (356), (456) } $, the corresponding factor graph looks like
            \begin{center}
            \tikzstyle{var}=[circle, draw, thick, minimum size=25pt, inner sep=0pt, fill=yellow]
            \tikzstyle{fun}=[rectangle, draw, thick, minimum size=25pt, inner sep=0pt, fill=cyan]
            \tikzstyle{edge} = [draw, thick, -]
            \begin{tikzpicture}[scale=1, auto, swap]
                \def\mysep{2}
                \foreach \i in {1,...,6} {
                    \node[var] (S\i) at (\i*\mysep, 0) {$ S_\i $};
                    \node[fun] (h\i) at (\i*\mysep, 0.75 * \mysep) {$ h_\i $};
                    \path[edge] (S\i) -- (h\i);
                }
                \foreach \i/\idx in {1/126, 2/234, 3/345, 4/456}
                    \node[fun] (J\idx) at (5/3*\i*\mysep-4/3, -1.5*\mysep) {$ J_{\idx} $};
                \foreach \source/ \dest in {  J126/S1, J126/S2, J126/S6,
                                              J234/S2, J234/S3, J234/S4,
                                              J345/S3, J345/S4, J345/S5,
                                              J456/S4, J456/S5, J456/S6}
                    \path[edge] (\source) -- (\dest);
            \end{tikzpicture}
            \end{center}
    \end{enumerate}
    \end{solution}
    

\item[(2)] {\bf Independent set problem}

    Independent set is a problem defined and studied in combinatorics and
    graph theory. Given a (unweighted, undirected) graph $G(V,E)$, an independent set $S \subseteq V$ is
    defined as a
    subset of nodes such that if $i \in S$ then for all $j
    \in \partial i$ we have $j \notin S$. In other words in for all $(ij)\in E$
    only $i$ or $j$ can belong to the independent set. 

    (a) Write a probability distribution that is uniform over all
    independent sets on a given graph. (b) Write a probability distribution
    that gives a larger weight to larger independent sets, where the
    size of an independent set is simply $|S|$. 

    \begin{solution} $\,$ 
    \begin{enumerate}[(a)]
    \item   
            First let's construct the factor graph for this problem.
            Denote $ N = \abs{V} $ as the number of nodes in the graph, we can use a length-$ N $ spin configuration $ \udl{\sigma}^S $ to represent any set $ S \subseteq V $ by
            \begin{equation*}
                \sigma_i^S = 
                \begin{cases}
                    +1, &\text{ if node } i \text{ is in } S\\
                    -1, &\text{ if node } i \text{ is not in } S\\
                \end{cases}
            \end{equation*}
            Besides, for any edge $ (ij) \in E $, we associated it with a function node $ e_{ij} $ whose compatibility function is $ \psi_{ij}(\sigma_i, \sigma_j) = \mathbb{I}(\sigma_i + \sigma_j < 2) $, which equals to $ 1 $ whenever $ i,j \in S $ and $ (ij) \in E $

            A node set $ S \subseteq V $ is an independent set if and only if $ \psi_{ij}(\sigma_i^S, \sigma_j^S) = 1 $ for all distinct $ i,j \in S $.
            The probability distribution that is uniform over all independent sets
            \begin{equation} \label{eq:2_1}
                P(\udl{\sigma})
                = \frac{1}{Z} \prod_{(ij) \in E} \psi_{ij}(\sigma_i, \sigma_j)
                = \frac{1}{Z} \prod_{(ij) \in E} \mathbb{I}(\sigma_i + \sigma_j < 2).
            \end{equation}
            
            For example, suppose we have the following graph
            \begin{center}
            \tikzstyle{vertex}=[circle, draw, thick, minimum size=25pt,inner sep=0pt]
            \tikzstyle{edge} = [draw, thick, -]
            \tikzstyle{weight} = [font=\small]
            \begin{tikzpicture}[scale=1.5, auto, swap]
                \def\mysep{0.5}
                \foreach \i/\x/\y in { 1/0/1.732, 2/-1/0, 3/1/0, 4/0/3.732, 5/-2.732/-1, 6/2.732/-1 }
                    \node[vertex] (S\i) at (\x*\mysep, \y*\mysep) {$ \i $};
                \foreach \source/ \dest in {1/2, 1/3, 2/3, 1/4, 2/5, 3/6}
                    \path[edge] (S\source) --  (S\dest);
            \end{tikzpicture}
            \end{center}
            The corresponding factor graph looks like
            \begin{center}
            \tikzstyle{var}=[circle, draw, thick, minimum size=25pt, inner sep=0pt, fill=yellow]
            \tikzstyle{fun}=[rectangle, draw, thick, minimum size=25pt, inner sep=0pt, fill=cyan]
            \tikzstyle{edge} = [draw, thick, -]
            \begin{tikzpicture}[scale=1, auto, swap]
                \def\mysep{2}
                \foreach \i in {1,...,6}
                    \node[var] (S\i) at (\i*\mysep, 0) {$ \sigma_\i $};
                \foreach \i/\idx in {1/12, 2/13, 3/23, 4/14, 5/25, 6/36}
                    \node[fun] (e\idx) at (\i*\mysep, -\mysep) {$ e_{\idx} $};
                \foreach \source/ \dest in {12/1, 12/2, 13/1, 13/3, 23/2, 23/3,
                                            14/1, 14/4, 25/2, 25/5, 36/3, 36/6}
                    \path[edge] (e\source) -- (S\dest);
            \end{tikzpicture}
            \end{center}
    \item   
            Note that $ \abs{S} = (N + \sum_{i=1}^N \sigma_i^S) / 2 $.
            If we want a probability distribution that gives a larger weight to larger independent sets, we can simply introduce a positive increasing function $ g(\cdot) $ and multiply $ g(\abs{S}) $ to the probability distribution in part (a), i.e.
            \begin{equation} \label{eq:2_2}
                P(\udl{\sigma})
                = g \rdbrs{ \frac{N + \sum_{i=1}^N \sigma_i}{2} } \times \frac{1}{Z} \prod_{(ij) \in E} \mathbb{I}(\sigma_i + \sigma_j < 2).
            \end{equation}
            For example, we can choose $ g(x) = \exp \rdbrs{ \mu x } $ for some $ \mu > 0 $.
            The last thing to notice is that the normalizing constant $ Z $ is different in eqn~\ref{eq:2_1} and eqn~\ref{eq:2_2}.
    \end{enumerate}
    \end{solution}

\item[(3)]{\bf Matching problem}

    Matching is another classical problem of graph theory. It is related
    to a dimer problem in statistical physics. Given a  (unweighted,
    undirected) graph $G(V,E)$ a
    matching $M \subseteq E$ is defined as a subset of edges such that if $(ij) \in M$ then
    no other edge that contains node $i$ or $j$ can be in $M$. In other
    words a matching is a subset of edges such that no two edges of the
    set share a node. 


    (a) Write a probability distribution that is uniform over all
    matchings on a given graph. (b) Write a probability distribution
    that that gives a larger weight to larger matchings, where the
    size of a matching is simply $|M|$. 

    \begin{solution} $\,$ 
    \begin{enumerate}[(a)]
    \item   
            First let's construct the factor graph for this problem.
            We can use a length-$ \abs{E} $ spin configuration $ \udl{\sigma}^M $ to represent any set $ M \subseteq E $ by
            \begin{equation*}
                \sigma_{ij}^M = 
                \begin{cases}
                    +1, &\text{ if edge } (ij) \text{ is in } M\\
                    -1, &\text{ if edge } (ij) \text{ is not in } M\\
                \end{cases}.
            \end{equation*}
            Besides, for any node $ i \in V $, we associated it with a function node $ f_i $ whose compatibility function is
            \begin{equation*}
                \psi_i(\clbrs{ \sigma_{ij} }_{j \in \partial i}) = \mathbb{I} \rdbrs{ \frac{1}{2} \sum_{j \in \partial i} (\sigma_{ij}+1) \le 1 },
            \end{equation*}
            which equals to $ 1 $ whenever there is at most one edge in $ M $ that is incident to node $ i $.

            An edge set $ M \subseteq E $ is a matching if and only if $ \psi_i(\clbrs{ \sigma_{ij}^M }_{j \in \partial i}) = 1 $ for all $ i \in S $.
            The probability distribution that is uniform over all matchings on a given graph
            \begin{equation}\label{eq:3_1}
                P(\udl{\sigma})
                = \frac{1}{Z} \prod_{i \in V} \psi_i(\clbrs{ \sigma_{ij} }_{j \in \partial i})
                = \frac{1}{Z} \prod_{i \in V} \mathbb{I} \rdbrs{ \sum_{j \in \partial i} (\sigma_{ij}+1) \le 2 }.
            \end{equation}

            For example, suppose we have the same graph as last independent set problem.
            The corresponding factor graph looks like
            \begin{center}
            \tikzstyle{var}=[circle, draw, thick, minimum size=25pt, inner sep=0pt, fill=yellow]
            \tikzstyle{fun}=[rectangle, draw, thick, minimum size=25pt, inner sep=0pt, fill=cyan]
            \tikzstyle{edge} = [draw, thick, -]
            \begin{tikzpicture}[scale=1, auto, swap]
                \def\mysep{2}
                \foreach \i/\idx in {1/12, 2/13, 3/23, 4/14, 5/25, 6/36}
                    \node[var] (e\idx) at (\i*\mysep, 0) {$ \sigma_{\idx} $};
                \foreach \i in {1,...,6}
                    \node[fun] (f\i) at (\i*\mysep, -\mysep) {$ f_\i $};
                \foreach \source/ \dest in {12/1, 12/2, 13/1, 13/3, 23/2, 23/3,
                                            14/1, 14/4, 25/2, 25/5, 36/3, 36/6}
                    \path[edge] (e\source) -- (f\dest);
            \end{tikzpicture}
            \end{center}
    \item   
            Note that $ \abs{M} = (\abs{E} + \sum_{(ij) \in E} \sigma_{ij}^M) / 2 $.
            If we want a probability distribution that gives a larger weight to larger matchings, we can simply introduce a positive increasing function $ g(\cdot) $ and multiply $ g(\abs{M}) $ to the probability distribution in part (a), i.e.
            \begin{equation} \label{eq:3_2}
                P(\udl{\sigma})
                = g \rdbrs{ \frac{\abs{E} + \sum_{(ij) \in E} \sigma_{ij}}{2} } \times \frac{1}{Z} \prod_{i \in V} \mathbb{I} \rdbrs{ \sum_{j \in \partial i} (\sigma_{ij}+1) \le 2 }.
            \end{equation}
            For example, we can choose $ g(x) = \exp \rdbrs{ \mu x } $ for some $ \mu > 0 $.
            The last thing to notice is that the normalizing constant $ Z $ is different in eqn~\ref{eq:3_1} and eqn~\ref{eq:3_2}.
    \end{enumerate}
    \end{solution}

\end{itemize}
\end{document}